\chapter{Esempio }
\begin{multicols}{2}
    Si sceglie come esempio \cite{kouWearableAllFabricHybrid2024} che offre una realizzazione pratica di un harvester ibrido. L'obbiettivo di questo generatore e' proporre un dispostivo riproducibile in scala e facilmente integrabile nel vestiario. Per fare questo usa una struttura interamente fondata su una base di tessuto e accoppia un generatore triboelettrico a uno RF. Un tessuto elettricamente conduttivo disponibile sul mercato e' usato per realizzare le connessioni su ambo i lati di uno strato di tessuto comune di cotone. In questo modo l'harvester diventa altamente flessibile e facilmente ingrabile in capi di abbigliamento. Avere 2 fonti distinte aumenta le probabilita' che il dispositvo sia sempre in funzione, inoltre queste in particolare si prestano a un design planare.

\begin{section}{Generatori}
    \begin{subsection}{Generatore Triboelettrico}
        In generale i generatori di questo tipo sfruttano la capacita' di alcuni materiali di trasferire carica al contatto. La struttura e' formata da uno strato di propilene fluorato (FEP) libero di scorrere sopra ai 2 elettrodi di tessuto connettivo che doppiano da secondo materiale triboelettrico. Questa forma detta a scorrimento libero, oltre ad avere un fattore di forma piu' vantaggioso, e' genera anche piu' energia rispetto al semplice contatto \cite{fuAchievingUltraDurabilityHigh2024}. Il design aperto e i limiti di peso non permettono di ottimizzare altri aspetti che influenzano l'efficienza, come l'umidita' o la pressione tra gli elementi.
        \# figura e spiegazione?
        La tensione massima ottenibile tra gli elettrodi e' determinata dalla densita' di carica ottenibile dal contatto dei materiali. . Lo stimolo meccanico genera nello strato libero un movimento oscillatorio e quindi una corrente alternata sul circuito che necessita rettifica. La densita' di potenza massima e' \(0.024\mu Wcm^{-2}\) stata misurata collegando un potenziometro e azionando il movimento a \(2Hz\) per simulare il cammino. Per quanto questo dimostri che il dispositivo e' in grado di raccogliere energia dal movimento umano, la densita' di potenza e' molto bassa se paragonata a una batteria tradizionale, ma anche ai requisiti correnti dei dispositivi raccolti in \cite{AdvancedEnergyHarvesters}.
    \end{subsection}
    
    \begin{subsection}{Generatore a Radiofrequenza}
        Per raccogliere energia dalle radiazioni presenti nell'ambiente e' stato scelto un harvester con antenna dimensionata per la sola banda attorno ai \(2.45GHz\) usata per connessioni a corto raggio come Wi-Fi e Bluetooth. La distribuzione di onde radio dovute alle comunicazioni elettroniche e' naturalmente concentrata nelle zone urbane. In aree extraurbane la densita' di potenza dovuta a comunicazioni ad ampio raggio scende vistosamente \cite{ibrahimRadioFrequencyEnergy2022}, la scelta di una banda comunemente usata da dispositivi generalmente vicini all'uomo ha quindi il vantaggio di essere piu' consistente nel tempo. La configurazione per l'antenna e' a patch circolare e le dimensioni ottimali per la risonanza sono stati ottenuti attraverso simulazione. Il valore di specific absorbtion rate (SAR) e' un parametro usato per determinare la potenza assorbita da un'unita' di massa di tessuto corporeo \cite{vallozzi26LatestDevelopments2016}. Secondo le simulazioni il SAR e' \(0.073\frac{W}{Kg}\), ben inferiore ai \(2.0\frac{W}{Kg}\) stabiliti dalle norme europee per antenne mobili. In questo caso il SAR e' usato come parametro di efficienza rispetto alla quantita' di radiazione persa dall'antenna nel corpo.
    \end{subsection}

    \begin{subsection}{Integrazione}
        
    \end{subsection}

\end{section}

\end{multicols}